\documentclass[12pt]{article}
% \usepackage[utf8]{inputenc}
% \usepackage[T1]{fontenc}
%% \usepackage{lmodern}
\usepackage{graphicx}
\usepackage[small]{caption}
\usepackage{amsmath, amssymb}
\usepackage{fullpage}
\newcommand{\R}{\mathbb{R}}
\newcommand{\Z}{\mathbb{Z}}
%helper macro for vectors
\newcommand{\vb}[1]{\ensuremath{\mathbf{#1}}}

\newtheorem{theorem}{Theorem}
\newtheorem{definition}{Definition}

\begin{document}
\author{Antoine Levitt}
\title{Finite volumes method for incompressible Navier-Stokes equations}
\maketitle
\abstract{TODO}
\tableofcontents
\newpage

\section{Introduction}
TODO
\section{Physical context}
We study a fluid contained in a d-dimensional (d = 2 or 3) domain
$\Omega$, with a boundary $\partial \Omega$. Positions are specified
by the d-dimensional vector \vb{x}. For every time $t \in \R^+$, the
fluid is described by its velocity $\vb{v}(\vb{x}, t)$, its density
$\rho(\vb{x}, t)$, its pressure $p(\vb{x},t)$ and its temperature
$T(\vb{x}, t)$. The model used are the Navier-Stokes equations in the
Boussinesq approximation, which lead to coupled partial differential
equations (PDEs) for these variables. Together with initial and
boundary condition, they specify the evolution of the flow.
\subsection{The Navier-Stokes equations}
The Navier-Stokes equations are a reformulation of conservation of
momentum (Newton's Second Law) and conservation of mass in the case of
a fluid. The conservation of momentum equation reads
\begin{equation}
  \label{ns-mom}
  \rho \left(\frac{\partial \mathbf{v}}{\partial t} + \mathbf{v} \cdot
    \nabla \mathbf{v}\right) = -\nabla p + \nabla \cdot\mathbb{T} +
  \mathbf{f}.
\end{equation}
$\mathbb{T}$ is the stress tensor, and \vb{f} are the external volumic
forces exerting on the fluid (gravity, electromagnetic ...).  We need
an expression of $\mathbb{T}$ as a function of the other variables for
the model to be complete. The second term, $\mathbf{v} \cdot \nabla
\mathbf{v}$, known as the convective acceleration, is to be understood
as $(\vb{v} \cdot \nabla) \vb{v}$, that is, the ``transport operator''
$(\vb{v} \cdot \nabla) = \sum_{i=1}^d v_i \frac{\partial}{\partial i}$
applied to $\vb{v}$.

The conservation of mass equation is
\begin{equation}
  \label{ns-mass}
  \frac{\partial \rho}{\partial t} + \nabla \cdot (\rho \mathbf{v}) = 0.
\end{equation}

\subsection{Natural convection and Boussinesq approximation}
We are now interested in studying natural convection. While in forced
convection the fluid flow is imposed (by boundary conditions either in
velocity or pression), in natural convection the fluid is set in
motion by temperature gradients alone. The Boussinesq approximation is
to neglect the variation of density everywhere in the Navier-Stokes
equations, except on the gravity term. This means that $\rho$ is
constant, equal to $\rho_0$, and the mass equation simply becomes
\begin{equation}
  \label{ns-b-mass}
  \nabla \vb{v} = 0.
\end{equation}
This is known as an incompressible flow.

We also assume the fluid to be Newtonian (which is the case of most
fluids commonly used, including the object of our study,
air). Together with incompressibility, this implies that $\nabla \cdot
\mathbb T = \mu \Delta \vb{v}$, where $\mu$ is the viscosity of the flow,
assumed to be constant.

The only external force here is gravity, so $\vb{f} = -\rho g
\vb{e_z}$. This time, $\rho$ is not assumed constant and depends on
the temperature $T$. At the first order, we can write $\rho - \rho_0 =
\alpha \delta T$, where $\delta T$ is the difference between the
actual temperature and a reference state. This leads to $\vb{f} =
(\rho_0 + \alpha \delta T) g \vb{e_z}$.

We then use the variable change $p = p + \rho_0 g z$, and $T = \delta
T$, which simplifies the equations. The momentum equation now becomes :

\begin{equation}
  \label{ns-mom}
  \rho \left(\frac{\partial \mathbf{v}}{\partial t} + \mathbf{v} \cdot
    \nabla \mathbf{v}\right) = -\nabla p + \mu \Delta \vb{v} + \alpha
  T \vb{e_z}.
\end{equation}

Since we introduced the variable $T$, we must add another equation so
as to describe the evolution of $T$. This is given by energy
conservation, which, under our hypotheses, is
\begin{equation}
  C_p \rho \left(\frac{\partial T}{\partial t} + \nabla \cdot (T \vb{v})\right) = k \Delta T
\end{equation}

Finally, we adimensionalize the equations by introducing Rayleigh and
Prandtl numbers $Ra$ and $Pr$ and normalizing $T$. The final form of our equations is then
\begin{align}
  \nabla \cdot \vb{v} &= 0& \text{Mass}\\
  \frac{\partial \mathbf{v}}{\partial t} + \mathbf{v} \cdot
    \nabla \mathbf{v} +\nabla p - Pr \Delta \vb{v} - Ra Pr T
  \vb{e_z} &= 0& \text{Momentum}\\
  \frac{\partial T}{\partial t} + \nabla \cdot (T \vb{v}) - \Delta
  T&=0& \text{Energy}
\end{align}

This is a system of $d + 2$ coupled PDEs with $d+2$ variables. We
solve it using a finite volumes scheme.

\section{Finite volumes}
\subsection{Theory}
\subsection{Implementation}
\section{The chimney problem}
\section{Numerical experiments}
\subsection{Methodology}
\subsection{Results}
\section{Conclusion}
TODO
\listoffigures


\begin{thebibliography}{99}
\end{thebibliography}
\end{document}
